\documentclass[a4paper]{jsarticle}
\usepackage[dvipdfmx]{graphicx}
\usepackage{amsmath}
\renewcommand{\thesection}{第\arabic{section}問}
\renewcommand{\thesubsection}{(\arabic{subsection})}
\renewcommand{\thesubsubsection}{(\alph{subsubsection})}
\begin{document}

\title{2024 分野1}
\author{nakao}
\maketitle

\section{}
\subsection{}
\subsubsection{}
部材の中立軸に直交する平面は,変形後においても
平面を保ち,中立軸に垂直である.
\subsubsection{}
変形後の部材の扇形がなす頂角を$\Delta \theta$として,
中立軸から高さ$z$離れた部分の変形後の長さを$\Delta s$とすると,
\begin{equation}
  \label{eq:delta_s}
  \Delta s = (R - z) \Delta \theta
\end{equation}
が成り立つ.
中立面では軸ひずみが生じないため
$\Delta s = dx$となるから
\begin{equation}
  \Delta \theta = \frac{dx}{R}
\end{equation}
である.これを式(\ref{eq:delta_s})に代入して,
\begin{equation}
  \Delta s = \left(1 - \frac{z}{R}\right) dx
\end{equation}
が得られる.
これより,軸ひずみ$\varepsilon$は
\begin{align}
  \varepsilon & = \frac{\Delta s - dx}{dx} = -\frac{z}{R} \\
\end{align}
となる.

\subsubsection{}
ヤング率を$E$とすると,軸応力$\sigma$は
\begin{equation}
  \sigma = E \varepsilon
  = -\frac{Ez}{R}
\end{equation}
となる.

\subsection{}
断面上の微小面積$\mathrm{d}A$に作用する軸力によるモーメントを断面全体で積分すると,
\begin{equation}
  \label{eq:moment}
  \begin{aligned}
    M & = \int_A (-z) \sigma \mathrm{d}A                       \\
      & = \int_A (-z) \left(–\frac{E z}{R}\right) \mathrm{d} A \\
      & = \frac{E}{R} \int_A z^2 \mathrm{d} A
    = \frac{E I}{R}
  \end{aligned}
\end{equation}
となる.

\subsection{}
\subsubsection{}
せん断応力が作用する面積は,
$2\left(\sqrt{r_2^2 - z^2} - \sqrt{r_1^2 - z^2}\right) dx$であり,
$x$軸方向の力のつり合いを考えると,
\begin{equation}
  \int_{\hat{A}} \left(\sigma +
  \frac{\mathrm{d} \sigma}{\mathrm{d} x} d x\right) \mathrm{d}A
  - \int_{\hat{A}} \sigma \mathrm{d} A
  - \tau \times 2\left(\sqrt{r_2^2 - z^2} - \sqrt{r_1^2 - z^2}\right) dx = 0
\end{equation}
が成り立つ.これより,
\begin{equation}
  \tau = \frac{1}{2\left(\sqrt{r_2^2 - z^2} - \sqrt{r_1^2 - z^2}\right)}
  \int_{\hat{A}}
  \frac{\mathrm{d} \sigma}{\mathrm{d} x} \mathrm{d}A
\end{equation}
が得られる.

\subsubsection{}
式(\ref{eq:moment})を$x$で微分して,
\begin{equation}
  \frac{\mathrm{d} M}{\mathrm{d} x} =
  EI \frac{\mathrm{d}}{\mathrm{d} x}
  \left[\frac{1}{R}\right]
\end{equation}
が得られる.ここで,
\begin{equation}
  \begin{aligned}
    \int_{\hat{A}} \frac{\mathrm{d} \sigma}{\mathrm{d} x} \mathrm{d} A
     & = \int_{\hat{A}} \frac{\mathrm{d}}{\mathrm{d} x}
    \left[-\frac{E z}{R}\right] \mathrm{d} A                   \\
     & = \int_{\hat{A}} (-E z) \frac{\mathrm{d}}{\mathrm{d} x}
    \left[\frac{1}{R}\right] \mathrm{d} A                      \\
     & = -E \frac{\mathrm{d}}{\mathrm{d} x}
    \left[\frac{1}{R}\right] \int_{\hat{A}} z \mathrm{d} A     \\
     & = - E \frac{1}{EI} \frac{\mathrm{d} M}{\mathrm{d} x} J  \\
     & = -\frac{J}{I} \frac{\mathrm{d} M}{\mathrm{d} x}
  \end{aligned}
\end{equation}
である.したがって,
\begin{equation}
  \tau = \frac{1}{2\left(\sqrt{r_2^2 - z^2} - \sqrt{r_1^2 - z^2}\right)}
  \int_{\hat{A}}
  \frac{\mathrm{d} \sigma}{\mathrm{d} x} \mathrm{d}A =
  -\frac{J}{2 I\left(\sqrt{r_2^2 - z^2} - \sqrt{r_1^2 - z^2}\right)}
  \frac{\mathrm{d} M}{\mathrm{d} x}
\end{equation}
となる.

\subsubsection{}
\begin{equation}
  \label{eq:first_moment}
  \begin{aligned}
    J & = \int_{\hat{A}} z \mathrm{d} A                        \\
      & = 2\left(
    \int_{-r_2}^{-r_1} \int_{0}^{\sqrt{r_2^2 - z^2}}
    z \mathrm{d} y \mathrm{d} z +
    \int_{-r_1}^{-z} \int_{\sqrt{r_1^2 - z^2}}^{\sqrt{r_2^2 - z^2}}
    z \mathrm{d} y \mathrm{d} z \right)                        \\
      & = 2 \left(
    \int_{-r_2}^{-r_1} z \sqrt{r_2^2 - z^2} \mathrm{d} z +
    \int_{-r_1}^{-z} z \left(\sqrt{r_2^2 - z^2} - \sqrt{r_1^2 - z^2}\right)
    \mathrm{d} y \mathrm{d} z \right)                          \\
      & = 2 \left(
    \int_{-r_2}^{-z} z \sqrt{r_2^2 - z^2} \mathrm{d} z -
    \int_{-r_1}^{-z} z \sqrt{r_1^2 - z^2} \mathrm{d} z \right) \\
      & = -\frac{2}{3}
    \left\{(r_2^2 - z^2)^{\frac{3}{2}} -
    (r_1^2 - z^2)^{\frac{3}{2}}\right\}
  \end{aligned}
\end{equation}
であるから,
\begin{equation}
  \begin{aligned}
    \tau & = -\frac{J}{2 I (\sqrt{r_2^2 - z^2} - \sqrt{r_1^2 - z^2})}
    \frac{\mathrm{d} M}{\mathrm{d} x}                                             \\
         & = \frac{(r_2^2 - z^2)^{\frac{3}{2}} - (r_1^2 - z^2)^{\frac{3}{2}}}
    {3 I (\sqrt{r_2^2 - z^2} - \sqrt{r_1^2 - z^2})}
    \frac{\mathrm{d} M}{\mathrm{d} x}                                             \\
         & = \frac{(r_2^2 - z^2) + \sqrt{r_2^2 - z^2} \sqrt{r_1^2 - z_2} +
      (r_1^2 - z^2)}
    {3I} \frac{\mathrm{d} M}{\mathrm{d} x}                                        \\
         & = \frac{r_1^2 + r_2^2 - 2 z^2 + \sqrt{(r_2^2 - z^2)(r_1^2 - z^2)}}{3I}
    \frac{\mathrm{d} M}{\mathrm{d} x}
  \end{aligned}
\end{equation}
と表せる.
これを$z$で微分すると
\begin{equation}
  \frac{\mathrm{d} \tau}{\mathrm{d} z} =
  -\frac{z}{3I} \frac{\mathrm{d} M}{\mathrm{d} x}
  \left(4 + \frac{r_1^2 + r_2^2 - 2z^2}{\sqrt{(r_2^2 - z^2)(r_1^2 - z^2)}}\right)
\end{equation}
となり,$-r_1 \leq z \leq r_1$の範囲では,$z = 0$で$\tau$が最大になる.
よって最大の$\tau$は
\begin{equation}
  \tau = \frac{r_1^2 + r_2^2 + r_1 r_2}{3 I}\frac{\mathrm{d}M}{\mathrm{d}x}
\end{equation}
となる.さらに,
\begin{equation}
  \begin{aligned}
    I & = \int_A z^2 \mathrm{d}A                       \\
      & = \int_0^{2\pi} \int_{r_1}^{r_2}
    (r \sin \theta)^2 r \mathrm{d} r \mathrm{d} \theta \\
      & = \int_{r_1}^{r_2} r^3 \mathrm{d} r
    \int_0^{2 \pi} \sin^2 \theta \mathrm{d} \theta     \\
      & = \frac{\pi(r_2^4 - r_1^4)}{4}
  \end{aligned}
\end{equation}
であるから,最大の$\tau$は
\begin{equation}
  \tau = \frac{4(r_1^2 + r_2^2 + r_1 r_2)}{3 \pi (r_2^4 - r_1^4)}
  \frac{\mathrm{d} M}{\mathrm{d} x}
\end{equation}
と表せる.

\subsection{}
\subsubsection{}
内側と外側の筒の断面二次モーメントをそれぞれ
$I_1, I_2$とすると,
\begin{align}
  I_1 & = \int_0^{2\pi} \int_{r_1}^{r_2} z^2 r \mathrm{d} r \mathrm{d} \theta
  = \frac{\pi (r_2^4 - r_1^4)}{4}                                             \\
  I_2 & = \int_0^{2\pi} \int_{r_2}^{r_3} z^2 r \mathrm{d} r \mathrm{d} \theta
  = \frac{\pi (r_3^4 - r_2^4)}{4}
\end{align}
曲率半径が$R$のときに
内側と外側のはりに生じる曲げモーメントをそれぞれ
$M_1, M_2$とすると,
\begin{align}
  M_1 & = \frac{E I_1}{R} = \frac{\pi E (r_2^4 - r_1^4)}{4 R} \\
  M_2 & = \frac{E I_2}{R} = \frac{\pi E (r_3^4 - r_2^4)}{4 R}
\end{align}
したがって,内側の外側のはりに生じるモーメントの割合はそれぞれ,
\begin{align}
  \frac{M_1}{M_1 + M_2} & = \frac{r_2^4 - r_1^4}{r_3^4 - r_1^4} \\
  \frac{M_2}{M_1 + M_2} & = \frac{r_3^4 - r_2^4}{r_3^4 - r_1^4} \\
\end{align}
である.

\subsubsection{}
No.
固着される前の状態で,
トータルの曲げモーメントが$M$であるとき,
\begin{equation}
  M = M_1 + M_2 = \frac{\pi E (r_3^4 - r_1^4)}{4 R}
\end{equation}
より,
\begin{equation}
  \label{eq:r_before}
  R = \frac{\pi E (r_3^4 - r_1^4)}{4 M}
\end{equation}
である.
固着された後の状態の断面二次モーメントを$I_3$とすると,
\begin{equation}
  I_3 = \int_0^{2\pi} \int_{r_1}^{r_3} z^2 r \mathrm{d} r \mathrm{d} \theta
  = \frac{\pi (r_3^4 - r_1^4)}{4}                                             \\
\end{equation}
であり,曲率半径が$R^{\prime}$のときに生じる曲げモーメントは
\begin{equation}
  \frac{EI_3}{R^{\prime}} = \frac{\pi E (r_3^4 - r_1^4)}{4 R^{\prime}}
\end{equation}
となる.曲げモーメントが$M$のときに対応する$R^{\prime}$は,
\begin{equation}
  \label{eq:r_after}
  R^{\prime} = \frac{\pi E (r_3^4 - r_1^4)}{4 M}
\end{equation}
となる.
式(\ref{eq:r_before}), (\ref{eq:r_after})
の比較より,同一の曲げモーメント$M$のもとでの曲率半径は固着する前後で等しい.

\subsection{}
No.
内径が$r_{\mathrm{in}}$,外径が$r_{\mathrm{out}}$の筒
の微小な長さ$\Delta x$の部分を考える.
$-r_{\mathrm{in}} \leq z \leq r_{\mathrm{in}}$において,
このに生じるせん断力を$V$として,力のつり合い
\begin{equation}
  \int_{\hat{A}} \left(
  \sigma + \frac{\mathrm{d} \sigma}{\mathrm{d} x} \Delta x
  \right) \mathrm{d} A -
  \int_{\hat{A}} \sigma \mathrm{d} A
  - V = 0
\end{equation}
より,
\begin{equation}
  V  = \int_{\hat{A}} \frac{\mathrm{d} \sigma}{\mathrm{d} x}
  \Delta x \mathrm{d} A
  = -E \Delta x \frac{\mathrm{d}}{\mathrm{d} x}
  \left[\frac{1}{R}\right] \int_{\hat{A}} z \mathrm{d} A
\end{equation}
が得られる.式(\ref{eq:first_moment})と同様に計算すると,
\begin{equation}
  V = \frac{2 E \Delta x}{3} \frac{\mathrm{d}}{\mathrm{d} x}
  \left[\frac{1}{R}\right]
  \left\{(r_{\mathrm{out}}^2 - z^2)^{\frac{3}{2}}
  - (r_{\mathrm{in}}^2 - z^2)^{\frac{3}{2}}\right\}
\end{equation}
となる.
\begin{equation}
  \frac{\partial V}{\partial z} =
  -2 E \Delta x \frac{\mathrm{d}}{\mathrm{d} x}
  \left[\frac{1}{R}\right]
  \frac{z (r_{\mathrm{out}}^2 - r_{\mathrm{in}}^2)}
  {\sqrt{r_{\mathrm{out}}^2 - z^2} + \sqrt{r_{\mathrm{in}}^2 - z^2}}
\end{equation}
より,$V$は$r_{\mathrm{in}}, r_{\mathrm{out}}$の値によらず
$z = 0$で最大値をとり,その値は,
\begin{equation}
  V = \frac{2 E \Delta x}{3} \frac{\mathrm{d}}{\mathrm{d} x}
  \left[\frac{1}{R}\right] (r_{\mathrm{out}}^3 - r_{\mathrm{in}}^3)
\end{equation}
となる.\par
同一の変形,すなわち同一の$R$のもとで,
はりが固着する前の最大のせん断力は,
\begin{equation}
  \label{eq:shear_before}
  \frac{2 E \Delta x}{3} \frac{\mathrm{d}}{\mathrm{d} x}
  \left[\frac{1}{R}\right] (r_{2}^3 - r_{1}^3) +
  \frac{2 E \Delta x}{3} \frac{\mathrm{d}}{\mathrm{d} x}
  \left[\frac{1}{R}\right] (r_{3}^3 - r_{2}^3) =
  \frac{2 E \Delta x}{3} \frac{\mathrm{d}}{\mathrm{d} x}
  \left[\frac{1}{R}\right] (r_{3}^3 - r_{1}^3)
\end{equation}
はりを固着した後の最大のせん断力は,
\begin{equation}
  \label{eq:shear_after}
  \frac{2 E \Delta x}{3} \frac{\mathrm{d}}{\mathrm{d} x}
  \left[\frac{1}{R}\right] (r_{3}^3 - r_{1}^3)
\end{equation}
式(\ref{eq:shear_before}),(\ref{eq:shear_after})の比較より,
固着する前後ではりに生じるせん断力の最大値は等しい.

\section{}
\subsection{}
\subsubsection{}
\begin{equation}
  w = a_1(t) \sin \frac{\pi x}{L}
\end{equation}
のとき,
\begin{align}
  \dot{w}          & = \dot{a}_1(t) \sin \frac{\pi x}{L}              \\
  w^{\prime\prime} & = -\frac{\pi^2}{L^2} a_1(t) \sin \frac{\pi x}{L}
\end{align}
が成り立つ.これを用いて,
\begin{equation}
  \begin{aligned}
    T & = \frac{1}{2} \int_0^L m \dot{w}^2 \mathrm{d} x \\
      & = \frac{1}{2} m \left(\dot{a}_1 (t)\right)^2
    \int_0^L \sin^2 \frac{\pi x}{L} \mathrm{d} x        \\
      & = \frac{m L}{4} \left(\dot{a}_1 (t)\right)^2
  \end{aligned}
\end{equation}
および,
\begin{equation}
  \begin{aligned}
    V & = \frac{1}{2} \int_0^L
    E I \left(-\frac{\pi^2}{L^2} a_1(t) \sin \frac{\pi x}{L}\right)^2
    \sin^2 \frac{\pi x}{L} \mathrm{d} x
    - m g \int_0^L a_1(t) \sin \frac{\pi x}{L} \mathrm{d} x \\
      & = \frac{E I \pi^4}{2 L^4}
    \left(a_1(t)\right)^2 \int_0^L \sin^2 \frac{\pi x}{L} \mathrm{d} x
    - m g a_1(t) \int_0^L \sin \frac{\pi x}{L} \mathrm{d} x \\
      & = \frac{E I \pi^4}{4 L^3} \left(a_1(t)\right)^2
    - \frac{2 m g L}{\pi} a_1(t)
  \end{aligned}
\end{equation}
が得られる.
$a_1(t)$に摂動$\delta a_1(t)$を加えると,
\begin{equation}
  Q_1 \delta a_1(t) =
  \int_0^L f(x, t) \delta a_1(t) \sin \frac{\pi x}{L} \mathrm{d} x
\end{equation}
が成り立つ.右辺を計算すると,
\begin{equation}
  \begin{aligned}
    \int_0^L f(x, t) \delta a_1(t) \sin \frac{\pi x}{L} \mathrm{d} x
     & = \int M g \delta (x - vt) \delta a_1(t) \sin \frac{\pi x}{L} \mathrm{d} x \\
     & = M g \delta a_1(t) \sin \frac{\pi v t}{L}
  \end{aligned}
\end{equation}
であるから,
\begin{equation}
  Q_1 = M g \sin \frac{\pi v t}{L}
\end{equation}
である.
\subsubsection{}
\begin{align}
  \frac{\mathrm{d}}{\mathrm{d} t}
  \left(\frac{\partial T}{\partial \dot{a}}\right) & =
  \frac{\mathrm{d}}{\mathrm{d} t}
  \left(\frac{m L}{2} \dot(a)(t)\right) =
  \frac{1}{2} m L \ddot{a}(t)                            \\
  \frac{\partial T}{\partial a_1}                  & = 0 \\
  \frac{\partial V}{\partial a_1}                  & =
  \frac{E I \pi^4}{2 L^3} a_1(t) - \frac{2 m g L}{\pi}
\end{align}
より,運動方程式は,
\begin{equation}
  \frac{1}{2} m L \ddot{a}_1 (t)
  + \frac{E I \pi^4}{2 L^3} a_1(t)
  - \frac{2 m g L}{\pi}
  - M g \sin \frac{\pi v t}{L} = 0
\end{equation}
となる.

\subsubsection{}
この系の固有振動数$\omega_0$は,
\begin{equation}
  \omega_0 = \sqrt{\frac{E I \pi^4}{2 L^3} \frac{m L}{2}} =
  \frac{\pi^2}{L^2} \sqrt{\frac{E I}{m}}
\end{equation}
である.系の固有周期が載荷の周期と一致するための条件は,
\begin{equation}
  \frac{l}{v} =  \frac{2 \pi}{\omega_0}
\end{equation}
すなわち,
\begin{equation}
  v = \frac{\pi l}{2 L^2} \sqrt{\frac{E I}{m}}
\end{equation}
と表せる.

\subsection{}
\subsubsection{}
$w(x, y) = 0$のときの高さを
$y = 0$として,鉛直上向きに$y$軸をとる.
質点の位置$\boldsymbol{r}(t)$は,
\begin{equation}
  \boldsymbol{r}(t) =
  \begin{pmatrix}
    vt \\ -w(vt ,t)
  \end{pmatrix} =
  \begin{pmatrix}
    vt \\ -a_1(t) \sin \frac{\pi v t}{L}
  \end{pmatrix}
\end{equation}
であるから,質点の速度は,
\begin{equation}
  \dot{\boldsymbol{r}}(t) =
  \begin{pmatrix}
    v \\
    -\dot{a}_1(t) \sin \frac{\pi v t}{L} -\frac{\pi v}{L} a_1(t) \cos \frac{\pi v t}{L}
  \end{pmatrix}
\end{equation}
と表せる.
質点の運動エネルギーと位置エネルギーを考慮すると,
\begin{equation}
  \begin{aligned}
    T & = \frac{1}{2} \int_0^L m \dot{w}^2 \mathrm{d} x
    + \frac{1}{2} M \left\| \dot{r} \right\|^2          \\
      & = \frac{mL}{4} \left(\dot{a}_1(t)\right)^2
    + \frac{M v^2}{2}
    + \frac{M}{2}
    \left(\dot{a}_1(t) \sin \frac{\pi v t}{L}
    + \frac{\pi v}{L} a_1(t) \cos \frac{\pi v t}{L}\right)^2
  \end{aligned}
\end{equation}
また,
\begin{equation}
  \begin{aligned}
    V & = \frac{1}{2} \int_0^L E I w^{\prime\prime} \mathrm{d} x
    - m g \int_0^L w \mathrm{d} x
    + M g \left(-a_1(t) \sin \frac{\pi v t}{L}\right)            \\
      & = \frac{E I \pi^4}{4 L^3} \left(a_1(t)\right)^2
    - \frac{2 m g L}{\pi} a_1(t)
    - M g a_1(t) \sin \frac{\pi v t} {L}
  \end{aligned}
\end{equation}
が得られる.

\subsubsection{}
\begin{align}
  \frac{\mathrm{d}}{\mathrm{d} t}
  \left(\frac{\partial T}{\partial \dot{a}}\right)
   & = \left\{\frac{m L}{2} + \frac{M}{2}
  \left(1 - \cos \frac{2 \pi v t}{L}\right)\right\} \ddot{a}_1(t)
  + \frac{3 \pi M v}{2 L} \dot{a}_1(t) \sin \frac{2 \pi v t}{L}
  L \frac{\pi^2 M v^2}{L^2} \cos \frac{2 \pi v t}{L} a_t(t)             \\
  \frac{\partial T}{\partial a_1}
   & = \frac{\pi M v}{2 L} \sin \frac{2 \pi v t}{L} \dot{a}_1(t)
  + \frac{\pi^2 M v^2}{2 L^2} \left(1 + \cos \frac{2 \pi v t}{L}\right) \\
  \frac{\partial V}{\partial a_1}
   & = \frac{\pi^4 E I}{2 L^3} a_1(t)
  - \frac{2 m g L}{\pi} - M g \sin \frac{\pi v t}{L}
\end{align}
より,運動方程式は,
\begin{equation}
  \begin{aligned}
    \left\{\frac{m L}{2} \right. & + \left.\frac{M}{2}
    \left(1 - \cos \frac{2 \pi v t}{L}\right)\right\} \ddot{a}_1(t)
    + \frac{\pi M v}{L} \sin \frac{2 \pi v t}{L} \dot{a}_1(t)                                   \\
                                 & + \left\{\frac{\pi^4 E I}{2 L^3} - \frac{\pi^2 M v^2}{2 L^2}
    \left(1 - \cos \frac{2 \pi v t}{L}\right)\right\} a_1(t)
    - \frac{2 m g L}{\pi} - M g \sin \frac{\pi v t}{L} = 0
  \end{aligned}
\end{equation}
となる.

\subsubsection{}
(2)(b)の運動方程式では,振動数$2 \pi v/L$の振動を表す非線形項があり,
これにより,車の通過で駆動される振動数の2倍の振動数をもつ高周波応答が生じる.
\end{document}