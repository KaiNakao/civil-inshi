\documentclass[a4paper]{jsarticle}
\usepackage[dvipdfmx]{graphicx}
\usepackage{amsmath}
\renewcommand{\thesection}{第\arabic{section}問}
\renewcommand{\thesubsection}{(\arabic{subsection})}
\renewcommand{\thesubsubsection}{(\alph{subsubsection})}
\begin{document}

\title{2024 分野1}
\author{nakao}
\maketitle

\section{}
\subsection{}
\subsubsection{}
部材の中立軸に直交する平面は,変形後においても
平面を保ち,中立軸に垂直である.
\subsubsection{}
変形後の部材の扇形がなす頂角を$\Delta \theta$として,
中立軸から高さ$z$離れた部分の変形後の長さを$\Delta s$とすると,
\begin{equation}
  \label{eq:delta_s}
  \Delta s = (R - z) \Delta \theta
\end{equation}
が成り立つ.
中立面では軸ひずみが生じないため
$\Delta s = dx$となるから
\begin{equation}
  \Delta \theta = \frac{dx}{R}
\end{equation}
である.これを式(\ref{eq:delta_s})に代入して,
\begin{equation}
  \Delta s = \left(1 - \frac{y}{R}\right) dx
\end{equation}
が得られる.
これより,軸ひずみ$\varepsilon$と軸応力$\sigma$は
\begin{align}
  \varepsilon & = \frac{\Delta s - dx}{dx} = -\frac{y}{R} \\
  \sigma      & = -E \varepsilon = - \frac{E y}{R}
\end{align}
となる.
\end{document}