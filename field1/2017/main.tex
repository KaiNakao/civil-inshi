\documentclass[a4paper]{jsarticle}
\usepackage[dvipdfmx]{graphicx}
\usepackage{amsmath}
\usepackage{bm}
\renewcommand{\thesection}{第\arabic{section}問}
\renewcommand{\thesubsection}{(\arabic{subsection})}
\renewcommand{\thesubsubsection}{(\alph{subsubsection})}
\begin{document}

\title{2017分野1}
\author{nakao}
\maketitle

\section{}
\subsection{}
つり合い式の
\begin{equation}
  E I w^{\prime \prime \prime \prime} = p
\end{equation}

\subsection{}
両端において、変位と曲げモーメントが0であるから、
\begin{align}
  w(0) &= 0 \\
  w(L) &= 0 \\
  w^{\prime \prime}(0) &= 0 \\
  w^{\prime \prime}(L) &= 0
\end{align}
である。

\subsection{}
\subsubsection{}
両端において、変位とたわみ角が0であるから、
\begin{align}
  w(0) &= 0 \\
  w(L) &= 0 \\
  w^{\prime}(0) &= 0 \\
  w^{\prime}(L) &= 0
\end{align}
である。

\subsubsection{}
鉛直方向の力のつり合いと、$x = L/2$に関する対称性より明らか。

\subsubsection{}
固定支持で載荷した構造系は、単純支持で載荷した構造系と、
単純支持で載荷せず両端に集中モーメントのみを加えた構造系の足し合わせとして表せる。
単純支持で載荷せず両端に集中モーメントのみを加えた構造系における変位が、
固定支持の場合と単純支持の場合の変位の差に相当している。

\section{}
\subsection{}
運動方程式は、
\begin{equation}
  \begin{cases}
    M \ddot{x_1} + k_1 (x_1 - x_2) &= f_1 \\
    m \ddot{x_2} + k_1 (x_2 - x_1) + k_2 x_2 &= f_2 \\
  \end{cases}
\end{equation}
であり、行列表示すると、
\begin{equation}
  \begin{pmatrix}
    M & 0 \\
    0 & m
  \end{pmatrix}
  \begin{pmatrix}
    \ddot{x_1} \\
    \ddot{x_2}
  \end{pmatrix} +
  \begin{pmatrix}
    k_1 & -k_1 \\
    -k_1 & k_1 + k_2
  \end{pmatrix}
  \begin{pmatrix}
    x_1 \\
    x_2
  \end{pmatrix} =
  \begin{pmatrix}
    f_1 \\
    f_2
  \end{pmatrix}
\end{equation}
となる。

\subsection{}
$M = 3m, k_2 = 3 k_1$より、運動方程式は
\begin{equation}
  \begin{pmatrix}
    3m & 0 \\
    0 & m
  \end{pmatrix}
  \begin{pmatrix}
    \ddot{x_1} \\
    \ddot{x_2}
  \end{pmatrix} +
  \begin{pmatrix}
    k_1 & -k_1 \\
    -k_1 & 4k_1
  \end{pmatrix}
  \begin{pmatrix}
    x_1 \\
    x_2
  \end{pmatrix} =
  \begin{pmatrix}
    f_1 \\
    f_2
  \end{pmatrix}
\end{equation}
と簡略化できる。ここで、行列$M, K$を
\begin{align}
  M &= 
  \begin{pmatrix}
    3m & 0 \\
    0 & m
  \end{pmatrix} \\
  K &=
  \begin{pmatrix}
    k_1 & -k_1 \\
    -k_1 & 4k_1
  \end{pmatrix}
\end{align}
とする。固有モード$\boldsymbol{\phi}$と固有振動数$\omega$について、
\begin{equation}
  (K - \omega^2 M) \boldsymbol{\phi} = \boldsymbol{0}
\end{equation}
が成り立つ。
$\boldsymbol{\phi} \neq \boldsymbol{0}$であるような$\boldsymbol{\phi}$が存在するとき、
\begin{equation}
  \det (K - \omega^2 M) = 0
\end{equation}
が満たされる。
\begin{equation}
  \det (K - \omega^2 M) = 
  \det
  \begin{pmatrix}
    k_1 - 3 m \omega^2 & -k_1 \\
    -k_1 & 4k_1 - m \omega^2
  \end{pmatrix} =
  3 m^2 \omega^4 - 13 k_1 m \omega^2 + 3 k_1^2
\end{equation}
であり、これを0とすると、
\begin{equation}
  \omega = \sqrt{\frac{(13 \pm \sqrt{133}) k_1}{6m}}
\end{equation}
を得る。これをもとに、
\begin{align}
  \omega_1 &= \sqrt{\frac{(13 + \sqrt{133}) k_1}{6m}} \\
  \omega_2 &= \sqrt{\frac{(13 - \sqrt{133}) k_1}{6m}} \\
\end{align}
とする。
$\omega_1, \omega_2$に対応する固有モードを
$\boldsymbol{\phi}_1, \boldsymbol{\phi}_2$として、
$(K - \omega_1^2 M) \boldsymbol{\phi}_1 = \boldsymbol{0}$を解くと、
\begin{equation}
  \boldsymbol{\phi}_1 \propto
  \begin{pmatrix}
    1 \\
    \frac{-11 - \sqrt{133}}{2}
  \end{pmatrix}
\end{equation}
を得る。
また、$(K - \omega_2^2 M) \boldsymbol{\phi}_2 = \boldsymbol{0}$を解くと、
\begin{equation}
  \boldsymbol{\phi}_2 \propto
  \begin{pmatrix}
    1 \\
    \frac{-11 + \sqrt{133}}{2}
  \end{pmatrix}
\end{equation}
を得る。
したがって、固有振動数は
$\omega_1 = \sqrt{\frac{(13 + \sqrt{133}) k_1}{6m}}, \omega_2 = \sqrt{\frac{(13 - \sqrt{133}) k_1}{6m}}$
であり、それぞれに対応する固有モードは、
$\boldsymbol{\phi}_1 = \left(1, \frac{-11-\sqrt{133}}{2}\right),\boldsymbol{\phi}_2 = \left(1, \frac{-11+\sqrt{133}}{2}\right)$
である。

\subsection{}
曲げの伝達が表現されないこと。減衰が考慮されないこと。
\end{document}