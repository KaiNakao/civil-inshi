\documentclass[a4paper]{jsarticle}
\usepackage[dvipdfmx]{graphicx}
\usepackage{amsmath}
\usepackage{bm}
\renewcommand{\thesection}{第\arabic{section}問}
\renewcommand{\thesubsection}{(\arabic{subsection})}
\renewcommand{\thesubsubsection}{(\alph{subsubsection})}
\begin{document}

\title{2019分野1}
\author{nakao}
\maketitle

\section{}
\subsection{}
$x$軸方向の力のつり合いから、
\begin{equation}
  (\sigma + \mathrm{d} \sigma) w \mathrm{d} z
  + (\tau + \mathrm{d} \tau) w \mathrm{d} x
  - \sigma w \mathrm{d} z
  + \tau w \mathrm{d} x = 0
\end{equation}
が成り立つ。式(1)を$w \mathrm{d} x \mathrm{d} z$で割ると、
\begin{equation}
  \frac{\mathrm{d} \sigma}{\mathrm{d} x}
  + \frac{\mathrm{d} \tau}{\mathrm{d} z} = 0
\end{equation}
となり、したがって、$x$軸方向のつり合い式は、
\begin{equation}
  \frac{\partial \sigma}{\partial{d} x}
  + \frac{\partial \tau}{\partial{d} z} = 0
\end{equation}
で与えられる。

\subsection{}
微小部分の右端まわりのモーメントのつり合いから、
\begin{equation}
  M + \mathrm{d} M - M - V \mathrm{d} x = 0
\end{equation}
となる。これより、
\begin{equation}
  V = \frac{\mathrm{d} M}{\mathrm{d} x}
\end{equation}
が成り立つ。

\subsection{}
\begin{align}
  V &= \iint \tau \mathrm{d} A \\
  M &= \iint \sigma z \mathrm{d} A
\end{align}
より、
\begin{align}
  V &= \int_{-\frac{h}{2}}^{\frac{h}{2}} \tau w \mathrm{d} z \\
  M &= \int_{-\frac{h}{2}}^{\frac{h}{2}} \sigma z w \mathrm{d} z
\end{align}
である。

\subsection{}
\begin{equation}
  \frac{\partial}{\partial x} [A(x) z] = A^{\prime}(x) z
\end{equation}
であるから、式(3)は
\begin{equation}
  \frac{\partial \tau}{\partial z} = -A^{\prime} (x) z
\end{equation}
と書き直せる。これより、
\begin{equation}
  \tau = \int \left(- A^{\prime}(x) z \right) \mathrm{d} z
  = -\frac{1}{2} A^{\prime}(x) z^2 + C(x)
\end{equation}
を得る。ここで、$C(x)$は任意の$x$の関数である。
境界条件$\tau(z = \pm h/2) = 0$より、
\begin{equation}
  C(x) = \frac{1}{2} A^{\prime}(x) \frac{h^2}{4}
\end{equation}
と定まり、
\begin{equation}
  \tau = \frac{1}{2} A^{\prime}(x) \left(\frac{h^2}{4} - z^2\right)
\end{equation}
となる。

\subsection{}
式(8)に式(14)、式(9)に$\sigma = A(x) z$をそれぞれ代入すると、
\begin{align}
  M &= \int_{-\frac{h}{2}}^{\frac{h}{2}} A(x) z^2 w \mathrm{d} z
  = \frac{A(x)}{12} w h^3 \\
  V &= \int_{-\frac{h}{2}}^{\frac{h}{2}}
  \frac{1}{2} A^{\prime}(x) \left(\frac{h^2}{4} - z^2\right) \mathrm{d} z
  = \frac{A^{\prime}(x)}{12} w h^3
\end{align}
となり、式(5)の成立が確認できる。

\subsection{}
せん断歪を0と近似した上で、直応力を部材全体にわたって求めることで$\sigma = A(x) z$と表したときの$A(x)$が得られる。
それと式(14)を用いてせん断歪を計算すれば良い。
\end{document}