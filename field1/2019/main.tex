\documentclass[a4paper]{jsarticle}
\usepackage[dvipdfmx]{graphicx}
\usepackage{amsmath}
\usepackage{bm}
\renewcommand{\thesection}{第\arabic{section}問}
\renewcommand{\thesubsection}{(\arabic{subsection})}
\renewcommand{\thesubsubsection}{(\alph{subsubsection})}
\begin{document}

\title{2019分野1}
\author{nakao}
\maketitle

\section{}
\subsection{}
$x$軸方向の力のつり合いから、
\begin{equation}
  (\sigma + \mathrm{d} \sigma) w \mathrm{d} z
  + (\tau + \mathrm{d} \tau) w \mathrm{d} x
  - \sigma w \mathrm{d} z
  + \tau w \mathrm{d} x = 0
\end{equation}
が成り立つ。式(1)を$w \mathrm{d} x \mathrm{d} z$で割ると、
\begin{equation}
  \frac{\mathrm{d} \sigma}{\mathrm{d} x}
  + \frac{\mathrm{d} \tau}{\mathrm{d} z} = 0
\end{equation}
となり、したがって、$x$軸方向のつり合い式は、
\begin{equation}
  \frac{\partial \sigma}{\partial{d} x}
  + \frac{\partial \tau}{\partial{d} z} = 0
\end{equation}
で与えられる。

\subsection{}
微小部分の右端まわりのモーメントのつり合いから、
\begin{equation}
  M + \mathrm{d} M - M - V \mathrm{d} x = 0
\end{equation}
となる。これより、
\begin{equation}
  V = \frac{\mathrm{d} M}{\mathrm{d} x}
\end{equation}
が成り立つ。

\subsection{}
\begin{align}
  V & = \iint \tau \mathrm{d} A     \\
  M & = \iint \sigma z \mathrm{d} A
\end{align}
より、
\begin{align}
  V & = \int_{-\frac{h}{2}}^{\frac{h}{2}} \tau w \mathrm{d} z     \\
  M & = \int_{-\frac{h}{2}}^{\frac{h}{2}} \sigma z w \mathrm{d} z
\end{align}
である。

\subsection{}
\begin{equation}
  \frac{\partial}{\partial x} [A(x) z] = A^{\prime}(x) z
\end{equation}
であるから、式(3)は
\begin{equation}
  \frac{\partial \tau}{\partial z} = -A^{\prime} (x) z
\end{equation}
と書き直せる。これより、
\begin{equation}
  \tau = \int \left(- A^{\prime}(x) z \right) \mathrm{d} z
  = -\frac{1}{2} A^{\prime}(x) z^2 + C(x)
\end{equation}
を得る。ここで、$C(x)$は任意の$x$の関数である。
境界条件$\tau(z = \pm h/2) = 0$より、
\begin{equation}
  C(x) = \frac{1}{2} A^{\prime}(x) \frac{h^2}{4}
\end{equation}
と定まり、
\begin{equation}
  \tau = \frac{1}{2} A^{\prime}(x) \left(\frac{h^2}{4} - z^2\right)
\end{equation}
となる。

\subsection{}
式(8)に式(14)、式(9)に$\sigma = A(x) z$をそれぞれ代入すると、
\begin{align}
  M & = \int_{-\frac{h}{2}}^{\frac{h}{2}} A(x) z^2 w \mathrm{d} z
  = \frac{A(x)}{12} w h^3                                         \\
  V & = \int_{-\frac{h}{2}}^{\frac{h}{2}}
  \frac{1}{2} A^{\prime}(x) \left(\frac{h^2}{4} - z^2\right) \mathrm{d} z
  = \frac{A^{\prime}(x)}{12} w h^3
\end{align}
となり、式(5)の成立が確認できる。

\subsection{}
せん断歪を0と近似した上で、直応力を部材全体にわたって求めることで$\sigma = A(x) z$と表したときの$A(x)$が得られる。
それと式(14)を用いてせん断歪を計算すれば良い。

\section{}
\subsection{}
\subsubsection{}
水面から鉛直上方$l - \frac{m}{\rho \pi r^2}$の高さから測った円柱上面の高さを$u$とする。
円柱にかかる$y$方向の力は、
\begin{equation}
  -m g + \rho g \pi r^2 \left\{l - \left(l - \frac{m}{\rho \pi r^2}\right) - u\right\}
  = -\rho g \pi r^2 u
\end{equation}
となり、運動方程式は、
\begin{equation}
  m \ddot{u} = -\rho g \pi r^2 u
\end{equation}
で与えられる。

\subsubsection{}
円柱は単振動し、その周期は$2 \pi \sqrt{\frac{m}{\rho g \pi r^2}}$となる。

\subsection{}
\subsubsection{}
質点のつり合い位置からの変位を$x$と表す。質点の運動方程式が
\begin{equation}
  M \ddot{x} = -2 k x
\end{equation}
であるから、$x$の一般解は、定数$C_1, C_2$を用いて
\begin{equation}
  x = C_1 \cos \sqrt{\frac{2k}{M}} t  + C_2 \sin \sqrt{\frac{2k}{M}} t
\end{equation}
で与えられる。初期条件$x(0) = A, \dot{x}(0) = 0$より$C_1, C_2$を決定し、
\begin{equation}
  x = A \cos \sqrt{\frac{2k}{M}} t
\end{equation}
となる。この微分より、質点の速度と加速度はそれぞれ、
\begin{align}
  \dot{x}  & = -\sqrt{\frac{2k}{M}} A \sin \sqrt{\frac{2k}{M}} t \\
  \ddot{x} & = -\frac{2k}{M} A \cos \sqrt{\frac{2k}{M}} t
\end{align}
となる。

\subsection{}
質点の$y$軸方向の変位を$y$で表す。
それぞれのばねの伸びは、$\sqrt{L^2 + y^2} - L$であり、
弾性力の鉛直方向成分は
\begin{equation}
  k \left(\sqrt{L^2 + y^2} - L\right) \frac{y}{\sqrt{L^2 + y^2}}
  = k y \left(1 - \frac{L}{\sqrt{L^2 + y^2}}\right)
\end{equation}
で与えられる。したがって、質点の運動方程式は、
\begin{equation}
  M \ddot{y} = -2k y \left(1 - \frac{L}{\sqrt{L^2 + y^2}}\right)
\end{equation}
となる。

\subsection{}
$(1 + \delta)^{\alpha} = 1 + \alpha \delta$において、
$\delta = y^2 / L^2, \alpha = -1/2$として、
\begin{equation}
  \left(1 + \frac{y^2}{L^2}\right)^{-\frac{1}{2}}
  \simeq 1 - \frac{y^2}{2 L^2}
\end{equation}
が得られる。これを用いると、式(25)の右辺は、
\begin{equation}
  -2 k y \left(1 - \frac{1}{\sqrt{1 + \frac{y^2}{L^2}}}\right)
  = -2 k y \left\{1 - \left(1 - \frac{y^2}{2 L^2}\right)\right\}
  = - \frac{k y^3}{L^2}
\end{equation}
となる。したがって、運動方程式は、
\begin{equation}
  M \ddot{y} =  - \frac{k y^3}{L^2}
\end{equation}
で近似できる。\par
質点のポテンシャルが$y$の4次関数となることから、この運動と単振動を比べると、
振動の端では質点の加減速が激しく、振動中心付近では質点の速度変化が小さい。
\end{document}