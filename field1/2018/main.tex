\documentclass[a4paper]{jsarticle}
\usepackage[dvipdfmx]{graphicx}
\usepackage{amsmath}
\usepackage{bm}
\renewcommand{\thesection}{第\arabic{section}問}
\renewcommand{\thesubsection}{(\arabic{subsection})}
\renewcommand{\thesubsubsection}{(\alph{subsubsection})}
\begin{document}

\title{2018分野1}
\author{nakao}
\maketitle

\section{}
\subsection{}
$x = L/2$の集中荷重$P$は分布荷重$P \delta (x - L/2)$とみなせるため、
\begin{equation}
  E I w^{\prime \prime \prime \prime}(x) = P \delta (x - L/2)
\end{equation}
が成り立つ。

\subsection{}
梁の両端で変位と曲げモーメントが0になるから、
\begin{align}
  w(0) &= 0 \\
  w(L) &= 0 \\
  w^{\prime \prime}(0) &= 0 \\
  w^{\prime \prime}(L) &= 0
\end{align}
を満たせばよい。

\subsection{}
\subsubsection{}
梁の両端で変位とたわみ角が0になるから、
\begin{align}
  w(0) &= 0 \\
  w(L) &= 0 \\
  w^{\prime}(0) &= 0 \\
  w^{\prime}(L) &= 0
\end{align}
を満たせばよい。

\subsubsection{}
両端のせん断力については、梁の$x = L/2$に関する対称性と鉛直方向の力のつり合いより、
単純支持でも固定支持でも同じである。\par
両端の曲げモーメントについては、単純支持では境界条件から0であるが、
固定支持では単純支持の状態から両端に反力として集中モーメントが加わることで、たわみ角が0になると考え、
固定支持のときのほうが大きい。

\subsubsection{}
固定支持で載荷した構造系は、単純支持で載荷した構造系と、
単純支持で載荷せず両端に集中モーメントのみを加えた構造系の足し合わせとして表せる。
単純支持で載荷せず両端に集中モーメントのみを加えた構造系における変位が、
固定支持の場合と単純支持の場合の変位の差に相当している。

\subsection{}
大きくなる。式(1)にみるように、変位は断面二次モーメントに反比例しているから。

\section{}
\subsection{}
バネの自然の位置を基準にすると、質点の鉛直方向変位は$-m g / k + y(t)$と表せる。よって、
\begin{equation}
  m \ddot{y} = -m g - k \left(-\frac{mg}{k}+ y(t) - u_g(vt)\right)
\end{equation}
したがって、
\begin{equation}
  m \ddot{y} = -k\left(y(t) - u_g(vt)\right)
\end{equation}
が成り立つ。

\subsection{}
鉛直下向きを正として、
\begin{equation}
  f(t) = -k \left(-\frac{mg}{k} + y(t) - u_g(vt)\right)
  = mg -k \left(y(t) - u_g(vt)\right)
\end{equation}
である。

\subsection{}
運動方程式が
\begin{equation}
  m \ddot{y} = - k y
\end{equation}
となり、一般解は定数$A_1, A_2$を用いて
\begin{equation}
  y = A_1 \cos \omega t + A_2 \sin \omega t
\end{equation}
と表せる。
初期条件$y(0) = y_0, \dot{y}(0) = \dot{y}_0$を満たすように$A_1,A_2$を定め、
\begin{equation}
  y = y_0 \cos \omega t + \frac{\dot{y}_0}{w} \sin \omega t
\end{equation}
を得る。

\subsection{}
$x = vt$より、
\begin{equation}
  u_g(vt) = 
  \begin{cases}
    0 & (t < 0) \\
    -\sin \omega t & (0 \leq t \leq \frac{\pi}{\omega}) \\
    0 & (\frac{\pi}{\omega} < t)
  \end{cases}
\end{equation}
である。\par
$0 \leq t \leq \frac{\pi}{\omega}$のとき、
$y = C t \cos \omega t$を式(11)に代入すると$C = \omega / 2$を得る。
よって、
\begin{equation}
  y = \frac{1}{2} \omega t \cos \omega t
\end{equation}
は式(11)の特解である。
したがって、一般解は定数$B_1, B_2$を用いて
\begin{equation}
  y = \frac{1}{2} \omega t \cos \omega t + B_1 \cos \omega t + B_2 \sin \omega t
\end{equation}
と表せる。
初期条件$y(0) = y_0, \dot{y}(0) = \dot{y}_0$を満たすように$B_1,B_2$を定め、
\begin{equation}
  y = \frac{1}{2} \omega t \cos \omega t + y_0 \cos \omega t 
  + \left(\frac{\dot{y}_0}{\omega} - \frac{1}{2}\right) \sin \omega t
\end{equation}
を得る。ここで、$t = \pi / \omega$とすると、
\begin{align}
  y\left(\frac{\pi}{\omega}\right) &= -\frac{\pi}{2} - y_0 \\
  \dot{y}\left(\frac{\pi}{\omega}\right) &= -\dot{y}_0
\end{align}
である。\par
$\frac{\pi}{\omega} < t$のとき、$u_g(vt) = 0$より、一般解は定数$C_1, C_2$を用いて
\begin{equation}
  y = C_1 \cos \omega t + C_2 \sin \omega t
\end{equation}
と表せる。式(20),(21)の初期条件を満たすように$C_1,C_2$を定めると、
\begin{equation}
  y = \left(\frac{\pi}{2} + y_0\right) \cos \omega t + \frac{\dot{y}_0}{\omega} \sin \omega t
\end{equation}
を得る。
式(19),(23)より、
\begin{equation}
  y = 
  \begin{cases}
    y = \frac{1}{2} \omega t \cos \omega t + y_0 \cos \omega t 
    + \left(\frac{\dot{y}_0}{\omega} - \frac{1}{2}\right) \sin \omega t
    & \left(0 \leq t \leq \frac{\pi}{\omega}\right) \\
    y = \left(\frac{\pi}{2} + y_0\right) \cos \omega t + \frac{\dot{y}_0}{\omega} \sin \omega t
    & \left(\frac{\pi}{\omega} < t\right)
  \end{cases}
\end{equation}
である。
\end{document}