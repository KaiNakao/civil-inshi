\documentclass[a4paper]{jsarticle}
\usepackage[dvipdfmx]{graphicx}
\usepackage{amsmath}
\renewcommand{\thesection}{第\arabic{section}問}
\renewcommand{\thesubsection}{(\arabic{subsection})}
\renewcommand{\thesubsubsection}{(\alph{subsubsection})}
\begin{document}

\title{2023 分野1}
\author{nakao}
\maketitle

\section{}
\subsection{}
変形後の部材の扇形がなす頂角を$\Delta \theta$として,
中立軸から高さ$y$離れた部分の変形後の長さを$\Delta s$とすると,
\begin{equation}
  \label{eq:delta_s}
  \Delta s = (R - y) \Delta \theta
\end{equation}
が成り立つ.
中立面では軸ひずみが生じないため
$\Delta s = dx$となるから
\begin{equation}
  \Delta \theta = \frac{dx}{R}
\end{equation}
である.これを式(\ref{eq:delta_s})に代入して,
\begin{equation}
  \Delta s = \left(1 - \frac{y}{R}\right) dx
\end{equation}
が得られる.
これより,軸ひずみ$\varepsilon$と軸応力$\sigma$は
\begin{align}
  \varepsilon & = \frac{\Delta s - dx}{dx} = -\frac{y}{R} \\
  \sigma      & = -E \varepsilon = - \frac{E y}{R}
\end{align}
となる.

\subsection{}
はりに生じる弾性エネルギー$H$は,
\begin{equation}
  \begin{aligned}
    H & = \frac{1}{2} \int_V \sigma \varepsilon \mathrm{d} V    \\
      & = \frac{1}{2} \int_{-b/2}^{b/2} \int_{-h}^h \int_0^L
    \left(-\frac{E y}{R}\right) \left(-\frac{y}{R}\right)
    \mathrm{d} x \mathrm{d} y \mathrm{d} z                      \\
      & = \frac{E}{2} \int_{-b/2}^{b/2} \int_{-h}^h y^2
    \mathrm{d} y \mathrm{d} z \int_0^L \frac{\mathrm{d} x}{R^2} \\
      & = \frac{E I}{2} \int_0^L \frac{\mathrm{d} x}{R^2}
  \end{aligned}
\end{equation}
となる.

\subsection{}
$w = \alpha x^3 + \beta x^2$のとき,
\begin{equation}
  \frac{1}{R} = 6 \alpha x + 2 \beta
\end{equation}
であるから,
\begin{equation}
  \begin{aligned}
    H & = \frac{E}{2} I \int_0^L (6 \alpha x + 2 \beta)^2 \mathrm{d} x \\
      & = 2 E I L (3 \alpha^2 L^2 + 3 \alpha \beta L + \beta^2)
  \end{aligned}
\end{equation}
を得る.したがって,$\Delta H$は
\begin{equation}
  \Delta H = 2 E I L \left(3 \alpha^2 L^2 + 3 \alpha \beta + \beta^2\right)
  - P \left(\alpha L^3 + \beta L^2\right)
\end{equation}
と表される.$\alpha, \beta$による$H$の偏微分を$0$として
\begin{align}
  \frac{\partial \Delta H}{\partial \alpha} & =
  2 E I L \left(6 \alpha L^2 + 3 \beta L\right) - P L^3 = 0 \\
  \frac{\partial \Delta H}{\partial \beta}  & =
  2 E I L \left(3 \alpha L + 2 \beta\right) - P L^2 = 0
\end{align}
を解くと,
\begin{equation}
  \alpha = -\frac{P}{6EI}, \beta = \frac{PL}{2EI}
\end{equation}
を得る.

\subsection{}
\subsubsection{}
$u$は微分方程式
\begin{equation}
  \frac{\mathrm{d} u}{\mathrm{d} x} = \frac{P}{GA}
\end{equation}
に従う.この一般解は定数$C$を用いて
\begin{equation}
  u = \frac{P}{G A} x + C
\end{equation}
と表され,境界条件$u(0) = 0$より$C = 0$と決定される.
$G = E / \left\{2 (1 + \nu)\right\}$より,
\begin{equation}
  u = \frac{2 (1 + \nu) P}{E A} x
\end{equation}
となる.
\subsubsection{}
$x \sim L$のとき,
\begin{align}
  u & = \frac{2 (1 + \mu)P}{EA} x  \sim \frac{PL}{EA} =
  \frac{PL^3}{E} \frac{1}{L^2 A}                                               \\
  w & = -\frac{P}{6 E I} x^3 + \frac{P L}{2 E I} x^2  \sim \frac{P L^3}{E I} =
  \frac{P L^3}{E} \frac{1}{I}
\end{align}
のように見積もれる.
$h \ll L$を仮定すると
\begin{equation}
  I = \int y^2 \mathrm{d} A < \int h^2 \mathrm{d} A = h^2 A \ll L^2 A
\end{equation}
であることから,
\begin{equation}
  u \sim \frac{P L^3}{E} \frac{1}{L^2 A} \ll \frac{P L^3}{E} \frac{1}{I} \sim w
\end{equation}
がわかる.

\subsection{}
\subsubsection{}
回転バネの弾性エネルギーを考慮すると,
この系におけるトータルの弾性エネルギー$H^{\prime}$は
\begin{equation}
  H^{\prime} = \frac{1}{2}\int_V \sigma \varepsilon \mathrm{d} V
  + \frac{1}{2} k \left(\frac{\mathrm{d} w}{\mathrm{d} x}(0)\right)^2
\end{equation}
と表せる.
外力による仕事$W^{\prime}$は
回転バネを導入する前の系と同じく
\begin{equation}
  W^{\prime} = P w(L)
\end{equation}
である.
したがって,
\begin{equation}
  \Delta H^{\prime} = H^{\prime} - W^{\prime} =
  \frac{1}{2}\int_V \sigma \varepsilon \mathrm{d} V
  + \frac{1}{2} k \left(\frac{\mathrm{d} w}{\mathrm{d} x}(0)\right)^2
  - P w(L)
\end{equation}
と表せる.

\subsubsection{}
$w^{\prime} = \alpha^3 x^3 + \beta^2 x^2 + \gamma x$
と仮定する.$w$は$x$の微分方程式
\begin{equation}
  E I \frac{\mathrm{d}^4 w}{\mathrm{d} x^4} = 0
\end{equation}
に従うため$x$の3次までの多項式で表せ,
$x = 0$での境界条件$w(0) = 0$より
定数項はゼロである.
回転バネがある場合は,
$x = 0$での境界条件で回転角$w^{\prime}(0)$が
一般には0ではないため,$x$の1次の係数をパラメータ表示に加えた.

\subsubsection{}
このパラメータ設定のもとで$\Delta H^{\prime}$は,
\begin{equation}
  \Delta H^{\prime} =
  2 E I L \left(3 L^{2} \alpha^{2} + 3 L \alpha \beta + \beta^{2}\right) - L P \left(L^{2} \alpha + L \beta + \gamma\right) + \frac{\gamma^{2} k}{2}
\end{equation}
と計算できる.$\alpha, \beta, \gamma$による偏微分を$0$として
\begin{align}
  \frac{\partial \Delta H}{\partial \alpha} & =
  L^{2} \left(6 E I \left(2 L \alpha + \beta\right) - L P\right) = 0 \\
  \frac{\partial \Delta H}{\partial \beta}  & =
  L \left(2 E I \left(3 L \alpha + 2 \beta\right) - L P\right)= 0    \\
  \frac{\partial \Delta H}{\partial \gamma} & =
  -L P + \gamma k = 0
\end{align}
を解くと,
\begin{equation}
  \alpha = -\frac{P}{6 E I}, \beta = \frac{LP}{2EI},
  \gamma= \frac{LP}{k}
\end{equation}
が得られる.

\section{}
\subsection{}
1周分の$F_D$による仕事は,
\begin{equation}
  \begin{aligned}
    E_D & =
    \int_{0 \rightarrow 1} F_D \mathrm{d} x +
    \int_{2 \rightarrow 0} F_D \mathrm{d} x +
    \int_{0 \rightarrow 3} F_D \mathrm{d} x +
    \int_{4 \rightarrow 0} F_D \mathrm{d} x        \\ &=
    \int_{0}^{x_0} \eta k x \mathrm{d} x +
    \int_{x_0}^{0} (-\eta k x) \mathrm{d} x +
    \int_{0}^{-x_0} \eta k x \mathrm{d} x +
    \int_{-x_0}^{0} (-\eta k x) \mathrm{d} x       \\ &=
    \int_{0}^{x_0} \eta k x \mathrm{d} x +
    \int_{0}^{x_0} \eta k x \mathrm{d} x +
    \int_{-x_0}^{0} (-\eta k x) \mathrm{d} x +
    \int_{-x_0}^{0} (-\eta k x) \mathrm{d} x       \\ &=
    \frac{1}{2} \eta k \left(x_0\right)^2 \times 4 \\
        & = 2 \eta k \left(x_0\right)^2
  \end{aligned}
\end{equation}

\subsection{}
viscous dampingの係数を$c$として変位$x$は
\begin{equation}
  m \ddot{x} + c \dot{x} + k x = P_0 \sin (\omega t)
\end{equation}
に従う.この定常解は
\begin{align}
  x            & = \frac{P_0}{\sqrt{(-m \omega^2 + k)^2 + (c \omega)^2}}
  \sin (\omega t - \varphi)                                                          \\
  \sin \varphi & = \frac{-m \omega^2 + k}{\sqrt{(-m \omega^2 + k)^2 + (c \omega)^2}} \\
  \cos \varphi & = \frac{c \omega}{\sqrt{(-m \omega^2 + k)^2 + (c \omega)^2}}
\end{align}
と表され,
\begin{equation}
  \dot{x} = \frac{P_0 \omega}{\sqrt{(-m \omega^2 + k)^2 + (c \omega)^2}} \cos (\omega t - \varphi)
\end{equation}
が得られる.viscous dampingによる力$c \dot{x}$が
$t: \varphi/\omega \rightarrow (\varphi + 2 \pi) / \omega$の
一周期でする仕事は,
\begin{equation}
  \begin{aligned}
    \int c \dot{x} \mathrm{d} x
     & = \int_{\varphi / \omega}^{(\varphi + 2 \pi) / \omega}
    c \dot{x} \times \dot{x} \mathrm{d} t                                           \\
     & =\int_{\varphi / \omega}^{(\varphi + 2 \pi) / \omega}
    \frac{P_0 \omega c}{(-m \omega^2 + k)^2 + (c \omega)^2}
    \cos^2 (\omega t - \varphi) \mathrm{d} t                                        \\
     & = \frac{P_0 \omega c}{(-m \omega^2 + k)^2 + (c \omega)^2} \frac{\pi}{\omega} \\
     & = \frac{P_0 c \pi}{(-m \omega^2 + k)^2 + (c \omega)^2}
  \end{aligned}
\end{equation}
となる.これが$E_D$と等しくなるような$c$は,
\begin{equation}
  \label{eq:eta_to_c}
  c = \frac{2 \eta k \left(x_0\right)^2
    \left\{(-m \omega^2 + k)^2 + (c \omega)^2\right\}}{P_0 \pi}
\end{equation}

\subsection{}
静的に載荷したときの力のつり合いより,
\begin{equation}
  k X_{\mathrm{st}} = P_0
\end{equation}
が成り立つ.
equivalent viscous systemにおいて,
定常応答のピーク値が$3 X_{\mathrm{st}}$と等しくなる条件は,
\begin{equation}
  \frac{P_0}{\sqrt{(m \omega^2 + k)^2 + (c \omega)^2}} = 3 X_{\mathrm{st}}
\end{equation}
より,
\begin{equation}
  \sqrt{(m \omega^2 + k)^2 + (c \omega)^2} = \frac{k}{3}
\end{equation}
となる.これを満たす$c$は,
\begin{equation}
  c = \sqrt{-\frac{8k^2}{9 \omega^2} + 2 m k + m^2 \omega^2}
\end{equation}
これを式(\ref{eq:eta_to_c})に代入して,
\begin{equation}
  \eta = \frac{P_0 \pi}{2 k \left(x_0\right)^2
    \left\{(-m \omega^2 + k)^2 + (c \omega)^2\right\}}
  \sqrt{-\frac{8k^2}{9\omega^2} + 2mk + m^2 \omega^2}
\end{equation}
が得られる.

\subsection{}
\subsubsection{}
hysteretic dampingでは応答の振幅$x_0$が同じであれば,エネルギー散逸が周波数に依存しない.
viscous dampingでは,応答の振幅が同じ場合でも周波数によってエネルギー散逸が異なる.
\subsubsection{}
$P(t) = P_0 \sin (\omega t)$の外力を
$\omega = 0, \sqrt{k / m}$の2つの振動数で作用させ,
応答を観測する.
$w = 0$のときは
\begin{equation}
  x = \frac{P_0}{k},
\end{equation}
$w = \sqrt{k / m}$のときは
\begin{equation}
  x =\frac{P_0}{c}\sqrt{\frac{m}{k}} \sin (\omega t - \phi)
\end{equation}
が応答となり,$m, k$を既知として,これら振幅の比から$c$が推定できる.
\end{document}