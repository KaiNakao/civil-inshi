\documentclass[a4paper]{jsarticle}
\usepackage[dvipdfmx]{graphicx}
\usepackage{amsmath}
\usepackage{bm}
\renewcommand{\thesection}{第\arabic{section}問}
\renewcommand{\thesubsection}{(\arabic{subsection})}
\renewcommand{\thesubsubsection}{(\alph{subsubsection})}
\begin{document}

\title{2016分野1}
\author{nakao}
\maketitle

\section{}
\subsection{}
梁の中央に作用する集中荷重$P$は
分布荷重$P \delta(x - L/2)$と表せる。
したがって、
\begin{equation}
  E I w^{\prime \prime \prime \prime} = P \delta \left(x - \frac{L}{2}\right)
\end{equation}
が成り立つ。単純支持条件で梁の両端で鉛直方向の変位と曲げモーメントが0になるから、境界条件は
\begin{align}
  w(0) &= 0 \\
  w(L) &= 0 \\
  w^{\prime \prime}(0) &= 0 \\
  w^{\prime \prime}(L) &= 0
\end{align}
である。

\subsection{}
微小部分の力のつり合いより、曲げモーメントを$M(x)$とすると
\begin{equation}
  \frac{\mathrm{d}^2 M}{\mathrm{d} x^2} + P \delta \left(x - \frac{L}{2}\right) = 0
\end{equation}
が成り立つ。また、断面でのモーメントのつり合いから、
\begin{equation}
  M(x) = -E I(x) w^{\prime \prime}(x)
\end{equation}
が成り立つ。式(7)を式(6)に代入して、
\begin{equation}
  E \left(I(x) w^{\prime \prime \prime \prime}(x) + 
  2 I^{\prime}(x) w^{\prime \prime \prime}(x) + 
  I^{\prime \prime}(x) w^{\prime \prime}(x)\right)
  = P \delta \left(x - \frac{L}{2}\right)
\end{equation}
を得る。

\subsection{}
変化しない。
曲げモーメントは式(6)から求められ、この式は断面2次モーメントの分布によらない。

\subsection{}
変化する。
式(7)より、
\begin{equation}
  \frac{\mathrm{d}}{\mathrm{d} x} \frac{\mathrm{d} w}{\mathrm{d} x}
  = -\frac{E I(x)}{M(x)}
\end{equation}
である。
損傷の前後で$M(x)$は同一、$I(x)$のみが変化するが、式(9)より、たわみ角の微分が変化することになる。たわみ角が変化しないと仮定すると、たわみ角の微分も変化しないはずであり矛盾する。

\section{}
\subsection{}
\subsubsection{}
運動方程式は、
\begin{equation}
  m_1 \ddot{x}_1 + k_1 x_1 = -m \alpha
\end{equation}
となる。

\subsubsection{}
特解を$x = C_1 \sin \left(\sqrt{1.1} \omega_0 t\right)$
として運動方程式に代入すると、$C_1 = \frac{10}{\omega_0^2}$を得る。
したがって、解は
\begin{equation}
  x = \frac{10}{\omega_0^2} \sin \left(\sqrt{1.1} \omega_0 t\right) + C_2 \cos (\omega_0 t) + C_3 \sin (\omega_0 t)
\end{equation}
と表せる。初期条件を満たすように$C_2, C_3$を定め、
\begin{equation}
  x = \frac{10}{\omega_0^2} \sin \left(\sqrt{1.1} \omega_0 t\right)
  -\frac{10 \sqrt{1.1}}{\omega_0^2} \sin (\omega_0 t)
\end{equation}
を得る。

\subsection{}
\subsubsection{}
運動方程式は
\begin{equation}
  \begin{cases}
    m_1 \ddot{x}_1 + k_1 (x_1 - x_2) = -m_1 \alpha \\
    m_2 \ddot{x}_2 + k_1 (x_2 - x_1) + k_2 x_2 = -m_2 \alpha
  \end{cases}
\end{equation}
であり、これを行列表示すると、
\begin{equation}
  \begin{pmatrix}
    m_1 & 0 \\
    0 & m_2
  \end{pmatrix}
  \begin{pmatrix}
    \ddot{x}_1 \\
    \ddot{x}_2
  \end{pmatrix} +
  \begin{pmatrix}
    k_1 & -k_1 \\
    -k_1 & k_1 + k_2
  \end{pmatrix}
  \begin{pmatrix}
    x_1 \\
    x_2
  \end{pmatrix} =
  \begin{pmatrix}
    -m_1 \alpha \\
    -m_2 \alpha
  \end{pmatrix}
\end{equation}
である。

\subsubsection{}
行列$M,K$を
\begin{align}
  M &= 
  \begin{pmatrix}
    m_1 & 0 \\
    0 & m_2
  \end{pmatrix} \\
  K &= 
  \begin{pmatrix}
    k_1 & -k_1 \\
    -k_1 & k_1 + k_2
  \end{pmatrix}
\end{align}
とする。
このとき、固有振動数を$\omega$とおくと
$\det (K - \omega^2 M) = 0$が成り立つ。
\begin{equation}
  \begin{aligned}
    \det (K - \omega^2 M)
    &= \det
    \begin{pmatrix}
      k_1 - \omega^2 m_1 & -k_1 \\
      -k_1 & k_1 + k_2 - \omega^2 m_2
    \end{pmatrix} \\
    &= \det
    \begin{pmatrix}
      k_1 - \omega^2 m_1 & -k_1 \\
      -k_1 & 111 k_1 - 110 \omega^2 m_1
    \end{pmatrix} \\
    &= (k_1 - \omega^2 m_1) (111 k_1 - 110 \omega^2 m_1) - k_1^2 \\
    &= m^2 (10 \omega^2 - 11 \omega_0^2) (11 \omega^2 - 10 \omega_0^2)
  \end{aligned}
\end{equation}
であるから、固有振動数は
\begin{equation}
  \omega = \sqrt{\frac{10}{11}} \omega_0,
  \sqrt{\frac{11}{10}} \omega_0
\end{equation}
である。

\subsubsection{}
上で求めた固有振動数のうち、小さい方の$\sqrt{\frac{10}{11}} \omega_0$が
1次モードの固有振動数である。対応する固有ベクトルを$\boldsymbol{\phi}$として、
$(K - \omega^2 M) \boldsymbol{\phi} = \boldsymbol{0}$を解くと、
\begin{equation}
  \boldsymbol{\phi} \propto
  \begin{pmatrix}
    11 \\ 1
  \end{pmatrix}
\end{equation}
を得る。
したがって、固有振動において$x_1,x_2$の振幅をそれぞれ$A_1,A_2$とすると、
$A_1 / A_2 = 11$である。求める値は、
\begin{equation}
  \frac{A_1 - A_2}{A_2}
  = \frac{A_1}{A_2} - 1 = 10
\end{equation}
である。

\subsubsection{}
減衰の性能を高める。\par
剛性を高める。
\end{document}