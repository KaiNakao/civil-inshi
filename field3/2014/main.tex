\documentclass[a4paper]{jsarticle}
\usepackage[dvipdfmx]{graphicx}
\usepackage{amsmath}
\usepackage{bm}
\renewcommand{\thesection}{第\arabic{section}問}
\renewcommand{\thesubsection}{(\arabic{subsection})}
\renewcommand{\thesubsubsection}{(\alph{subsubsection})}
\begin{document}

\title{2014分野3}
\author{nakao}
\maketitle

\section{}
\subsection{}
式[1],[2]の左辺第1項は局所加速度であり、特定の場所における流速の変化率を表す。
第2,3項は移流加速度であり、移動する流体粒子の加速度を表す。

\subsection{}
流体の密度が一様であれば、
$x$方向、$z$方向の長さがそれぞれ$\Delta x, \Delta z$の微小領域において、
流出と流入の流量が一致するから、
\begin{equation}
  (u + \frac{\partial u}{\partial x} \Delta x - u) \Delta z
  + (w + \frac{\partial w}{\partial z} \Delta z - w) \Delta x = 0
\end{equation}
が成り立つ。両辺を$\Delta x \Delta z$で割ると、
\begin{equation}
  \frac{\partial u}{\partial x} + \frac{\partial w}{\partial z} = 0
\end{equation}
を得る。

\subsection{}
式(2)より、
\begin{equation}
  \begin{aligned}
    u \frac{\partial u}{\partial x} + w \frac{\partial u}{\partial z}
    &= u \frac{\partial u}{\partial x} + w \frac{\partial u}{\partial z}
    + u \left(\frac{\partial u}{\partial x} + \frac{\partial w}{\partial z}\right) \\
    &= \frac{\partial (uu)}{\partial x} + \frac{\partial (uw)}{\partial z}
  \end{aligned}
\end{equation}
であり、同様に、
\begin{equation}
  \begin{aligned}
    u \frac{\partial w}{\partial x} + w \frac{\partial w}{\partial z}
    &= u \frac{\partial w}{\partial x} + w \frac{\partial w}{\partial z}
    + w \left(\frac{\partial u}{\partial x} + \frac{\partial w}{\partial z}\right) \\
    &= \frac{\partial (uw)}{\partial x} + \frac{\partial (ww)}{\partial z}
  \end{aligned}
\end{equation}
が成り立つ。これを式[1],[2]に代入することで、式[3],[4]が得られる。

\subsection{}
$(x, \eta(t, x))$に存在した水粒子は時間$\Delta t$の間に
$(x + u \Delta t, \eta(t, x) + w \Delta t)$に移動するから、
\begin{equation}
  \eta(t + \Delta t, x + u \Delta t) = \eta(t, x) + w \Delta t
\end{equation}
が成り立つ。左辺をTaylor展開で1次近似すると、
\begin{equation}
  \eta(t + \Delta t, x + u \Delta t)
  = \eta(t, x) + \frac{\partial \eta}{\partial t} \Delta t
  + \frac{\partial \eta}{\partial x} u \Delta t
\end{equation}
であり、これを式(5)に代入して整理すると、式[5]が得られる。\par
同様に、$(x, z_b(x))$に存在した水粒子は時間$\Delta t$の間に
$(x + u \Delta t, z_b(x) + w \Delta t)$に移動するから、
\begin{equation}
  z_b(x + u \Delta t) = z_b(x) + w \Delta t
\end{equation}
が成り立つ。左辺をTaylor展開で1次近似すると、
\begin{equation}
  z_b(x + u \Delta t)
  = z_b(x) + \frac{\partial z_b}{\partial x} u \Delta t
\end{equation}
であり、これを式(7)に代入して整理すると、式[6]が得られる。

\subsection{}
式[2]と仮定より、
\begin{equation}
  \frac{\partial p}{\partial z}
  = -\rho \left(g + \frac{\partial w}{\partial t}
  + u \frac{\partial w}{\partial x} + w \frac{\partial w}{\partial z}\right)
  \simeq -\rho g
\end{equation}
である。これを$z$から$\eta$まで鉛直方向に積分すると、
\begin{equation}
  p_0 - p = -\rho g (\eta - z)
\end{equation}
となり、式[7]が得られる。å

\subsection{}

\end{document}