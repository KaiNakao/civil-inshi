\documentclass[a4paper]{jsarticle}
\usepackage[dvipdfmx]{graphicx}
\usepackage{amsmath}
\usepackage{bm}
\renewcommand{\thesection}{第\arabic{section}問}
\renewcommand{\thesubsection}{(\arabic{subsection})}
\renewcommand{\thesubsubsection}{(\alph{subsubsection})}
\begin{document}

\title{2016分野3}
\author{nakao}
\maketitle

\section{}
\subsection{}
\subsubsection{}
管路に作用する力を右向き正で$f$とすると、流体には$-f$の力が作用する。
断面$\mathrm{I}$,$\mathrm{I}\hspace{-1.2pt}\mathrm{I}$での流速をそれぞれ$v_1, v_2$、
断面$\mathrm{I}\hspace{-1.2pt}\mathrm{I}$(噴射直前)における圧力を$p_2$とすると、
運動量保存則から
\begin{equation}
  \rho Q v_2 - \rho Q v_1 = -f + p_1 A_1 - p_2 \frac{A_1}{4}
\end{equation}
が成り立つ。
ここで連続式
$Q = A_1 v_1 = \frac{A_1}{4} v_2$より、
\begin{align}
  v_1 &= \frac{Q}{A_1} \\
  v_2 &= \frac{4Q}{A_1}
\end{align}
である。また、Bernoulliの定理
\begin{equation}
  \frac{v_1^2}{2g} + \frac{p_1}{\rho g}
  = \frac{v_2^2}{2g} + \frac{p_2}{\rho g}
\end{equation}
より、
\begin{equation}
  p_2 = p_1 - \frac{\rho}{2} (v_2^2 - v_1^2)
  = p_1 - \frac{15 \rho Q^2}{2 A_1^2}
\end{equation}
である。式(2),(3),(5)を式(1)に代入すると、
\begin{equation}
  f = \frac{3 p_1 A_1}{4} - \frac{9 \rho Q^2}{8 A_1}
\end{equation}
を得る。

\subsubsection{}
断面$\mathrm{I}\hspace{-1.2pt}\mathrm{I}$(噴射直前)と
板に衝突した後の流れで運動量保存を考え、
\begin{equation}
  -\rho Q v_2 = -F + p_2 \frac{A_1}{4}
\end{equation}
が成り立つ。式(3),(5)をこれに代入して、
\begin{equation}
  F = \frac{p_1 A_1}{4} + \frac{17 \rho Q^2}{8 A_1}
\end{equation}
を得る。

\subsection{}
\subsubsection{}
容器から観測すると$-2g$の物体力が鉛直方向に作用する。水圧分布は図1のようになる。
\subsubsection{}
容器から観測すると物体力が作用しない。水圧分布は図2のようになる。
\begin{figure}[htb]
  \begin{minipage}{0.4\hsize}
    \centering
    \includegraphics[width=\hsize]{fig1.png}
    \caption{(a)の水圧分布}
  \end{minipage}
  \begin{minipage}{0.4\hsize}
    \centering
    \includegraphics[width=0.5\hsize]{fig2.png}
    \caption{(b)の水圧分布}
  \end{minipage}
\end{figure}

\subsection{}
鉛直方向の運動方程式は、
\begin{equation}
  \frac{\partial w}{\partial t}
  + u \frac{\partial w}{\partial x}
  + w \frac{\partial w}{\partial z}
  = - \frac{1}{\rho} \frac{\partial p}{\partial z} - g
\end{equation}
である。壁面付近では鉛直方向の流速が卓越し$u \ll w$であるとして、式(9)は
\begin{equation}
  \frac{\partial w}{\partial t}
  + w \frac{\partial w}{\partial z}
  = - \frac{1}{\rho} \frac{\partial p}{\partial z} - g
\end{equation}
で近似でき、これより
\begin{equation}
  \frac{\partial p}{\partial z} = -\rho g
  - \rho \left(\frac{\partial w}{\partial t}
  + w \frac{\partial w}{\partial z}\right)
\end{equation}
を得る。\par
点Aの水圧を$p_A$として、問題文の図5の状態の壁面における水面高さを$h$とする。
このとき、
\begin{equation}
  \int_0^h \frac{\partial p}{\partial z} \mathrm{d} z = 0 - p_A
\end{equation}
であるから、
\begin{equation}
  p_A = -\int_0^h \frac{\partial p}{\partial z} \mathrm{d} z
\end{equation}
と表せる。これに式(11)を代入すると、
\begin{equation}
  p_A = \int_0^h \left\{\rho g
  + \rho \left(\frac{\partial w}{\partial t}
  + w \frac{\partial w}{\partial z}\right)\right\} \mathrm{d} z
  = \rho g h + \rho \int_0^h
  \left(\frac{\partial w}{\partial t}
  + w \frac{\partial w}{\partial z}\right) \mathrm{d} z
\end{equation}
を得る。問題の図5の流速分布では$0 \leq z \leq h$で
$\frac{\partial w}{\partial t} > 0, \frac{\partial w}{\partial z} > 0$
であり、$p_A > \rho g h$となっている。
\end{document}