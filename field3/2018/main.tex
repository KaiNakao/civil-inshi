\documentclass[a4paper]{jsarticle}
\usepackage[dvipdfmx]{graphicx}
\usepackage{amsmath}
\usepackage{bm}
\renewcommand{\thesection}{第\arabic{section}問}
\renewcommand{\thesubsection}{(\arabic{subsection})}
\renewcommand{\thesubsubsection}{(\alph{subsubsection})}
\begin{document}

\title{2018分野3}
\author{nakao}
\maketitle

\section{}
\subsection{}
\subsection{}
\subsection{}
\subsection{}
\subsubsection{}
ベルヌーイの定理は以下の式で与えられる。
\begin{equation}
  \frac{\partial}{\partial s}
  \left(\frac{v^2}{2 g} + \frac{p}{\rho g} + z + h_f\right) = 0
\end{equation}
(変数の定義は一般的なものなので省略。。) \\
管の曲がりの部分では速度水頭が小さくなり、その分を圧力水頭や位置水頭の増加で補うことになる。
サージタンクがなければ位置水頭を増加させることができず圧力水頭が急増し、
圧力勾配が保たれず逆流する可能性がある。

\subsubsection{}
流出口での速度水頭が$\frac{v^2}{2g}$であることを考えると
ダム湖の水面と流出口における全水頭の差は$H - \frac{v^2}{2g}$である。
これとエネルギー保存則より、
\begin{equation}
  H - \frac{v^2}{2g} = K_e \frac{v^2}{2g} + f \frac{L}{d} \frac{v^2}{2g}
  + K_b \frac{v^2}{2g} + f \frac{L_2}{d} \frac{v^2}{2g} + H_T +
  f \frac{L_3}{d} \frac{v^2}{2g}
\end{equation}
を得る。したがって、
\begin{equation}
  H_T = H - \left(K_e + K_b + 1 + f\frac{L_1 + L_2 + L_3}{d}\right) \frac{v^2}{2g}
\end{equation}
となる。

\subsubsection{}
質量$m$の水塊の損失水頭$H_T$から取り出せるエネルギーは$mgH_T$である。
ここでは流量が$Q = \pi d^2 v$であり、有効落差$H_T$による仕事率$P$は、
\begin{equation}
  P = \rho Q g H_T = \rho g \pi d^2 v H_T
\end{equation}
と表せる。したがって、
\begin{equation}
  T = P \eta = \rho g \pi d^2 v H_T \eta
\end{equation}
となる。

\subsubsection{}
式(5)から$v$に依存する部分$v H_T$を最大化すれば良い。式(3)を用いると、
\begin{equation}
  \frac{\mathrm{d}}{\mathrm{d} v}(v H_T)
  = H - \frac{3}{2g} \left(K_e + K_b + 1 + f \frac{L_1 + L_2 + L_3}{d}\right) v^2
\end{equation}
である。これを0として、
\begin{equation}
  v = \sqrt{\frac{2 g H}{3 (K_e + K_b + 1 + f \frac{L_1 + L_2 + L_3}{d})}}
\end{equation}
で発電量が最大になる。

\section{}
\subsection{}
質量保存則は、
\begin{equation}
  \frac{\partial \rho}{\partial t}
  + \frac{\partial (\rho u)}{\partial x} 
  + \frac{\partial (\rho w)}{\partial z} = 0
\end{equation}
と表せる。流体の非圧縮性から
$\boldsymbol{\nabla} \rho = \boldsymbol{0},\frac{\partial \rho}{\partial t} = 0$であり、
\begin{equation}
  \frac{\partial u}{\partial x} 
  + \frac{\partial w}{\partial z} = 0
\end{equation}
となる。

\subsection{}
水平方向、鉛直方向の運動方程式はそれぞれ、
\begin{align}
  \frac{\partial u}{\partial t} + u \frac{\partial u}{\partial x} + w \frac{\partial u}{\partial z} &= -\frac{1}{\rho} \frac{\partial p}{\partial x} \\
  \frac{\partial w}{\partial t} + u \frac{\partial w}{\partial x} + w \frac{\partial w}{\partial z} &= -\frac{1}{\rho} \frac{\partial p}{\partial z} -g
\end{align}
で与えられる。式(10),(11)の両方において、左辺は流体の加速度を表す。
特に第1項は流速場の時間変化、第2項は流体粒子の移動による速度変化に対応する。
右辺は流体に作用する力を表す。ここでは圧力勾配による力と物体力としての重力を考慮している。

\subsection{}
\subsubsection{}
(11)より、
\begin{equation}
  \frac{\partial p}{\partial z}
  = -\rho \left(g + \frac{\partial u}{\partial t} + u \frac{\partial u}{\partial x} + w \frac{\partial u}{\partial z}\right)
\end{equation}
であるが、長波条件下では重力に比べて鉛直加速度が小さいと考え、
\begin{equation}
  \frac{\partial p}{\partial z} = -\rho g
\end{equation}
で近似できる。これを$z$から$\eta$まで鉛直方向に積分すると、
\begin{equation}
  p_a - p = -\rho g (\eta - z)
\end{equation}
となり、
\begin{equation}
  p = p_a + \rho g (\eta - z)
\end{equation}
を得る。
\subsubsection{}
式[2]に粘性抵抗の積分項
$\frac{\tau_{sx}}{\rho}, \frac{\tau_{bx}}{\rho}$
が残っていることから、ここでは、粘性抵抗まで考慮してみる。
粘性係数を$\mu$、水平面に作用するせん断応力を$\tau_x$とすると、
$\tau_x = \mu \frac{\partial u}{\partial z}$
であるから、$x$軸方向に作用する物体力は、
\begin{equation}
  -\frac{1}{\rho} \frac{\partial p}{\partial x}
  + \frac{\mu}{\rho} \frac{\partial^2 u}{\partial z^2}
  = -\frac{1}{\rho} \left(\frac{\partial p_a}{\partial x}
  + \rho g \frac{\partial \eta}{\partial x}\right)
  + \frac{1}{\rho} \frac{\partial \tau_x}{\partial z}
  = -\frac{1}{\rho} \frac{\partial p_a}{\partial x}
  - g \frac{\partial \eta}{\partial x}
  + \frac{1}{\rho} \frac{\partial \tau_x}{\partial z}
\end{equation}
と表せる。

\subsection{}
\subsubsection{}
定常状態では、$\frac{\partial Q}{\partial t} = 0$である。
また、式[1]を用いると、
\begin{equation}
  \frac{\partial}{\partial x} \frac{Q^2}{D}
  \simeq \frac{\partial}{\partial x} \frac{Q^2}{h}
  = \frac{2Q}{h} \frac{\partial Q}{\partial x}
  = -\frac{2Q}{h} \frac{\partial \eta}{\partial t} = 0
\end{equation}
がわかる。よって、式[2]は、
\begin{equation}
  -\frac{1}{\rho} 
  \int_{-h}^{\eta} \frac{\partial p}{\partial x} \mathrm{d} z
  + \frac{\tau_{sx} - \tau_{bx}}{\rho} = 0
\end{equation}
と簡略化できる。さらに、
\begin{equation}
  \frac{\partial p}{\partial x}
  = \frac{\partial p_a}{\partial x} + \rho g \frac{\partial \eta}{\partial x}
\end{equation}
で$z$に依存しないため、式(18)は
\begin{equation}
  -\frac{1}{\rho} 
  \left(\frac{\partial p_a}{\partial x}
  + \rho g \frac{\partial \eta}{\partial x}\right)
  (\eta + h)
  + \frac{\tau_{sx} - \tau_{bx}}{\rho} = 0
\end{equation}
と表せる。
ここに$\eta + h \simeq h$を代入して整理すると、
\begin{equation}
  \frac{\partial \eta}{\partial x}
  = \frac{1}{\rho g}
  \left(\frac{\tau_{sx} - \tau_{bx}}{h} - \frac{\partial p_a}{\partial x}\right)
\end{equation}
を得る。これを$x$で積分して、$\eta$は定数$C$を用いて
\begin{equation}
  \eta = \frac{1}{\rho g}
  \left(\frac{\tau_{sx} - \tau_{bx}}{h} x - p_a\right) + C
\end{equation}
と表せる。これに式[3]の気圧分布を代入し、
境界条件$\eta(0) = 0$を満たすように$C$を定めると、
\begin{equation}
  \eta = \frac{\tau_{sx} - \tau_{bx}}{h} x + 
  \frac{p_0 - p_L}{2 \rho g}
  \left\{1 - \cos \left(\frac{\pi}{L} x\right)\right\}
\end{equation}
を得る。ここで$x = L$を代入すると、堤防前面における水位は、
\begin{equation}
  \eta = \frac{\tau_{sx} - \tau_{bx}}{\rho g h} L + \frac{p_0 - p_L}{\rho g}
\end{equation}
となる。
\subsubsection{}
式(24)の第1項は$h$に反比例し、
$L$に比例する。ベンガル湾やメキシコ湾のように、湾内の水深が小さく、湾が長い場合には水位が大きくなりやすい。

\subsection{}
$\tau_{bx}$の絶対値が大きくなるとき、式(24)の第1項の値が小さくなり、
これは堤防前面における水位を下げる影響を及ぼす。


\end{document}